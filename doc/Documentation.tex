\documentclass{article}

\usepackage[utf8]{inputenc}
\usepackage{underscore}

\title{CDevStudio Documentation}
\author{Simon Wächter}

\begin{document}

\maketitle
\newpage

\tableofcontents
\newpage

\section{Introduction}
This developer guide gives an overview about the CDevStudio project, the requirements for it and some common work flow patterns.

\subsection{Idea of this documentation}
The idea is to document everything that is needed for this project in one document. With this documentation, a new person should be able to work for the project.

\subsection{Content of this documentation}
This documentation gives an overview about the design of the project and some common work flow patterns. For a code documentation check out the code documentation generated by Doxygen.

\section{Documentation}
There are two documentation for the CDevStudio project.

\subsection{Code documentation}
The code is documented with Doxygen. This documentation gives an overview about the codebase of the project, means classes, methods etc.

\subsection{Developer documentation}
As addition to the code documentation, there is a developer documentation. This documentation gives an overview about the design and implementation of the project. It does not contain code specific details.

\section{Codebase}
CDevStudio is based on a 3 layer structure. This means, the program is divided into a graphical (CDevStudio, CDevStudioCodeEdit), a business (CDevStudioPlatform) and a backend (CDevStudioBackend) layer.

\subsection{CDevStudio}
CDevStudio is the main program. It contains the graphical user interface and uses CDevStudioCodeEdit to display code. It accesses CDevStudioPlatform for project creation, lading, saving, editing etc.

\subsection{CDevStudioCodeEdit}
CDevStudioCodeEdit is able to display and highlight the code of a project.

\subsection{CDevStudioPlatform}
CDevStudioPlatform is the business layer of the project. The layer is able to create, load, delete and save projects. For file interaction, it uses the CDevStudioBackend layer.

\subsection{CDevStudioBackend}
CDevStudioBackend is the backend layer. It is responsible for I/O interaction.

\section{Work flow}
There are a few techniques that are needed for this project: A working toolchain and some Git knowledge.

\subsection{Requirements}
CDevStudio has a few requirements. You can split them into build and package requirements:

\begin{itemize}
	\item Build requirements
	\begin{itemize}
		\item Working C/C++ toolchain with support for the C++11 standard
		\item Git
		\item CMake (2.8.11 or higher)
		\item Qt 5 (5.1.0 or higher)
		\item Doxygen (Code documentation - Optional)
		\item Latex (Developer documentation - Optional)
	\end{itemize}
	\item Package requirements
	\begin{itemize}
		\item Linux Debian: build-essentials, dh_make, devscripts
		\item Linux Fedora: rpmbuild
		\item Linux Arch Linux: base-devel
		\item Linux Windows: NSIS
	\end{itemize}
\end{itemize}

\subsection{Initialize workspace}
For the workspace initialization, please clone the repository and create a new branch:

\begin{itemize}
	\item git clone http://github.com/swaechter/cdevstudio
	\item cd cdevstudio
	\item git checkout -b develop origin/develop
\end{itemize}

\subsection{Build project}
Now the project is initialized and you can start with a first build. CDevStudio uses CMake as build system. It can generate project files for different IDE's. To run CMake and build the project run these commands:

\begin{itemize}
	\item mkdir build
	\item cd build
	\item cmake ..
	\item make (or the build command of your toolchain)
	\item Copy the library (Depends on the platform)
\end{itemize}

\subsection{Create a new feature}
Now it's the time to create a new feature branch, add your feature and push it back to the develop branch:

\begin{itemize}
	\item git pull origin develop
	\item git checkout develop
	\item git merge feature-new-stuff
	\item git push
	\item git branch -d feature-new-stuff
\end{itemize}

\subsection{Create a new release}
There are a few things that have to be done for each new release:

\begin{itemize}
	\item Create a new release branch
	\begin{itemize}
		\item git checkout -b release-x.x.x develop
	\end{itemize}
	\item Generate documentation
	\begin{itemize}
		\item Run the doxygen documentation generation
		\item Update this documentation and create a PDF
	\end{itemize}
	\item Update codebase
	\begin{itemize}
		\item Update src/cdevstudio/data/desktop/cdevstudio.desktop
		\item Update src/cdevstudio/data/man/cdevstudio.1.gz
		\item Update src/cdevstudio/data/text/about_about.html
		\item Update src/cdevstudiocodeedit/CMakeLists.txt
		\item Update src/cdevstudioplatform/CMakeLists.txt
		\item Update src/cdevstudiobackend/CMakeLists.txt
	\end{itemize}
	\item Update Debian package
	\begin{itemize}
		\item Add new changelog entry
		\item Update package/linux_debian_deb/create_package.sh
	\end{itemize}
	\item Update Fedora package
	\begin{itemize}
		\item Update package/linux_fedora_rpm/cdevstudio.spec
		\item Update package/linux_fedora_rpm/create_package.sh
	\end{itemize}
	\item Update Windows package
	\begin{itemize}
		\item Update package/windows_exe/package.nsis
	\end{itemize}
	\item Integrate the new release
	\begin{itemize}
		\item git checkout master
		\item git merge release-x.x.x
		\item git push
		\item git checkout develop
		\item git merge release-x.x.x
		\item git push
		\item git branch -d release-x.x.x
	\end{itemize}
	\item Create the new release number
	\begin{itemize}
		\item git tag -a x.x.x -m "New release" master
		\item git push --tags
	\end{itemize}
\end{itemize}

\subsection{Create a package}
To create a package, run the create_package.sh or create_package.bat script in the package directory.

\end{document}
